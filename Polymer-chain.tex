\documentclass[a4paper,twocolumn]{article}
\usepackage{a4wide}
\usepackage[utf8]{inputenc}
\setlength{\columnsep}{0.8cm}
\usepackage{graphicx}
\usepackage{amsmath}
\usepackage{amssymb}
\usepackage{lipsum}
\begin{document}

%%%%%%% title
\title{Polymer-chains}
\date{}
\author{Ludwig Rasmijn\\ 4106644 \and Sebastiaan Lokhorst\\ 4005058 \and Shang-Jen Wang\\ 4215974}
\maketitle
%%%%%%%

\begin{abstract}
    Polymer-chains are simulated and a number of optimization are used to increase the length
\end{abstract}

\section{Introduction}
Rosenbluth algorithm and optimizations are used to simulate polymer-chains. 
 
\section{Methods}
The polymer model is a self-avoiding walk in 2 dimensions, where each effective monomer is a fixed distance from the nearest neighbour. The interaction between the beads are modelled by the Lennard-Jones potential.
\subsection{Rosenbluth algorithm}
The Rosenbluth algorithm will avoid high-energy conformations, when adding new beads to the polymer. This will avoid any unprobable conformations. It does this by adding the next bead with a distribution exp($-E(\theta)/(k_BT)$), where $E(\theta)$ is the interaction energy of the new bead and $\theta$ is the angle between the new bead and the previous two beads. This model uses six evenly spaced ($2\pi /6$) discrete angles with a random offset in which the new bead can be added to the polymer. The weight for each angle can be calculated by:
\begin{equation}\label{eq:weight}
    w_j^{(l)}=\text{exp}(-E(\theta_j)/(k_BT))\text{,}
\end{equation}\label{eq:sumweight}
where $j$ is the number of the angle and $l$ is the bead that is being added. The sum of all the weights can be expressed as:
\begin{equation}
    W^{(l)}=\sum_j \text{exp}(-E(\theta_j)/(k_BT))\text{.}
\end{equation}
So angle j is accepted with a probability $w_j^{(l)}/W^{(l)}$.\\
The polymer weight is given by:
\begin{equation}\label{eq:polweight}
    \text{PolWeight}=\prod_l W^l\text{.}
\end{equation}
\subsection{Pruned-enriched Rosenbluth method}
A method which suppresses the high energy conformations better than the Rosenbluth method is the pruned-enriched Rosenbluth method (PERM) by Grassberger \cite{grass}. This method removes the configurations with a `low' weight and replaces them with copies of configurations with a `high' weight. Two thresholds UpLim and LowLim are used to define `low' and `high' weights. If at any polymer length the PolWeight $>$ UpLim, then two members of this polymer are created when adding the next bead and each are given a PolWeight which half the PolWeight of the original polymer. This is called `enriching'. If at any polymer length the PolWeight $<$ LowLim, then the polymer will be removed with a probability of 1/2. If the polymer is not removed, then the PolWeight is multiplied by 2. This will make sure that the distribution does not change. This is called `pruning'.\\
The choice of UpLim and LowLim depends on the average weight `AvWeight' at step L. The average weight is updated for every polymer that reaches that length. UpLim and LowLim are expressed as a ratio
of the average weight and the weight `Weight3' corresponding to the shortest length (3 beads):
\begin{equation}\label{eq:uplim}
    \text{UpLim}=\alpha \cdot \text{AvWeight}/\text{Weight3} \text{,}
\end{equation}
\begin{equation}\label{eq:lowlim}
    \text{LowLim}=\alpha \cdot \text{AvWeight}/\text{Weight3} \text{.}
\end{equation}
A good value for $\alpha$ for UpLim is 2 and for LowLim 1.2. It is possible that $\alpha$ is dependent on L, but this can be removed by multiplying the weight of the polymer by a constant at each step. This constant should be near $1/(0.75N_{\theta})$, where $N_{\theta}$ is the number of angles. 
\begin{thebibliography}{9}

\bibitem{grass}
 P. Grassberger, (1997)\\
 $\textit{Pruned-enriched Rosenbluth method:} $\\ $\textit{Simulation of $\theta$ polymers of chain length}$ \\ $\textit{up to 1 000 000}$.\\
 Phys. Rev. E, 56, 3682–93
 

\end{thebibliography}
\end{document}